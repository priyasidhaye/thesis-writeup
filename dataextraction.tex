\chapter{Data Extraction}

Data was extracted from Twitter using the Twitter REST API using 51 search terms, or ‘hashtags’. These hashtags were chosen from a range of topics including pop culture,  international summit meetings discussing political issues, lawsuits and trials, social issues and health care issues like the recent outbreak of ebola. All these hashtags were ‘trending’ (being tweeted about at a high rate) at the time of extraction of the data. To give the data some variety, the data was extracted over the course of 15 days, which gave us multiple news stories to choose from for the search terms. Only English tweets were extracted since the study is limited to English. In the beginning, about 30,000 tweets were extracted, and more than half of these tweets, around 16,000 contained URLs referencing some news articles, photos on photo sharing sites, and videos. The hashtags were chosen to maximise the number of articles related to the tweets. Hence, a lot of topics that were chosen were being tweeted about by news agencies and other popular news sources.

The articles referenced by the tweets were extracted using the URLs mentioned in the tweets. The ‘newspaper’ package was used to extract article text and the title from the web page.