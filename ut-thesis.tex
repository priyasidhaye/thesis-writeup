                             %% ut-thesis.tex -- document template for graduate theses at UofT
%%
%% Copyright (c) 1998-2012 Francois Pitt <fpitt@cs.utoronto.ca>
%% last updated at 09:43 (EDT) on Fri  1 Jun 2012
%%
%% This work may be distributed and/or modified under the conditions of
%% the LaTeX Project Public License, either version 1.3c of this license
%% or (at your option) any later version.
%% The latest version of this license is in
%%     http://www.latex-project.org/lppl.txt
%% and version 1.3c or later is part of all distributions of LaTeX
%% version 2005/12/01 or later.
%%
%% This work has the LPPL maintenance status "maintained".
%%
%% The Current Maintainer of this work is
%% Francois Pitt <fpitt@cs.utoronto.ca>.
%%
%% This work consists of the files listed in the accompanying README.

%% SUMMARY OF FEATURES:
%%
%% All environments, commands, and options provided by the `ut-thesis'
%% class will be described below, at the point where they should appear
%% in the document.  See the file `ut-thesis.cls' for more details.
%%
%% To explicitly set the pagestyle of any blank page inserted with
%% \cleardoublepage, use one of \clearemptydoublepage,
%% \clearplaindoublepage, \clearthesisdoublepage, or
%% \clearstandarddoublepage (to use the style currently in effect).
%%
%% For single-spaced quotes or quotations, use the `longquote' and
%% `longquotation' environments.


%%%%%%%%%%%%         PREAMBLE         %%%%%%%%%%%%

%%  - Default settings format a final copy (single-sided, normal
%%    margins, one-and-a-half-spaced with single-spaced notes).
%%  - For a rough copy (double-sided, normal margins, double-spaced,
%%    with the word "DRAFT" printed at each corner of every page), use
%%    the `draft' option.
%%  - The default global line spacing can be changed with one of the
%%    options `singlespaced', `onehalfspaced', or `doublespaced'.
%%  - Footnotes and marginal notes are all single-spaced by default, but
%%    can be made to have the same spacing as the rest of the document
%%    by using the option `standardspacednotes'.
%%  - The size of the margins can be changed with one of the options:
%%     . `narrowmargins' (1 1/4" left, 3/4" others),
%%     . `normalmargins' (1 1/4" left, 1" others),
%%     . `widemargins' (1 1/4" all),
%%     . `extrawidemargins' (1 1/2" all).
%%  - The pagestyle of "cleared" pages (empty pages inserted in
%%    two-sided documents to put the next page on the right-hand side)
%%    can be set with one of the options `cleardoublepagestyleempty',
%%    `cleardoublepagestyleplain', or `cleardoublepagestylestandard'.
%%  - Any other standard option for the `report' document class can be
%%    used to override the default or draft settings (such as `10pt',
%%    `11pt', `12pt'), and standard LaTeX packages can be used to
%%    further customize the layout and/or formatting of the document.

%% *** Add any desired options. ***
\documentclass[doublespaced, 12pt]{ut-thesis}
\usepackage{natbib}
\usepackage{times}
\usepackage{url}
\usepackage{latexsym}
\usepackage{caption}
\usepackage{amsmath}

%
\usepackage{subfig}
\usepackage{lipsum}                    
\usepackage{xargs}                     
\usepackage{graphicx}
\graphicspath{ {images/} }
\usepackage[pdftex,dvipsnames]{xcolor} 
\usepackage[colorinlistoftodos,prependcaption,textsize=tiny]{todonotes}
\newcommandx{\unsure}[2][1=]{\todo[linecolor=red,backgroundcolor=red!25,bordercolor=red,#1]{#2}}
\newcommandx{\change}[2][1=]{\todo[linecolor=blue,backgroundcolor=blue!25,bordercolor=blue,#1]{#2}}
\newcommandx{\info}[2][1=]{\todo[linecolor=OliveGreen,backgroundcolor=OliveGreen!25,bordercolor=OliveGreen,#1]{#2}}
\newcommandx{\improvement}[2][1=]{\todo[linecolor=Plum,backgroundcolor=Plum!25,bordercolor=Plum,#1]{#2}}
\newcommandx{\thiswillnotshow}[2][1=]{\todo[disable,#1]{#2}}

\newcommand{\tabref}[1]{Table \ref{#1}}
\newcommand{\figref}[1]{Figure \ref{#1}}
\newcommand{\secref}[1]{Section \ref{#1}}
\newcommand{\chapref}[1]{Chapter \ref{#1}}


\newcommand{\limit}[0]{\textrm{limit}}

%% *** Add \usepackage declarations here. ***
%% The standard packages `geometry' and `setspace' are already loaded by
%% `ut-thesis' -- see their documentation for details of the features
%% they provide.  In particular, you may use the \geometry command here
%% to adjust the margins if none of the ut-thesis options are suitable
%% (see the `geometry' package for details).  You may also use the
%% \setstretch command to set the line spacing to a value other than
%% single, one-and-a-half, or double spaced (see the `setspace' package
%% for details).


%%%%%%%%%%%%%%%%%%%%%%%%%%%%%%%%%%%%%%%%%%%%%%%%%%%%%%%%%%%%%%%%%%%%%%%%
%%                                                                    %%
%%                   ***   I M P O R T A N T   ***                    %%
%%                                                                    %%
%%  Fill in the following fields with the required information:       %%
%%   - \degree{...}       name of the degree obtained                 %%
%%   - \department{...}   name of the graduate department             %%
%%   - \gradyear{...}     year of graduation                          %%
%%   - \author{...}       name of the author                          %%
%%   - \title{...}        title of the thesis                         %%
%%%%%%%%%%%%%%%%%%%%%%%%%%%%%%%%%%%%%%%%%%%%%%%%%%%%%%%%%%%%%%%%%%%%%%%%

%% *** Change this example to appropriate values. ***
\degree{Master of Science}
\department{Computer Science}
\gradyear{2016}
\author{Priya Sidhaye}
\title{Examining Indicative Tweet Generation As An Extractive Summarization Problem}

%% *** NOTE ***
%% Put here all other formatting commands that belong in the preamble.
%% In particular, you should put all of your \newcommand's,
%% \newenvironment's, \newtheorem's, etc. (in other words, all the
%% global definitions that you will need throughout your thesis) in a
%% separate file and use "\input{filename}" to input it here.


%% *** Adjust the following settings as desired. ***

%% List only down to subsections in the table of contents;
%% 0=chapter, 1=section, 2=subsection, 3=subsubsection, etc.
\setcounter{tocdepth}{2}

%% Make each page fill up the entire page.
\flushbottom


%%%%%%%%%%%%      MAIN  DOCUMENT      %%%%%%%%%%%%

\begin{document}

%% This sets the page style and numbering for preliminary sections.
\begin{preliminary}

%% This generates the title page from the information given above.
\maketitle

%% There should be NOTHING between the title page and abstract.
%% However, if your document is two-sided and you want the abstract
%% _not_ to appear on the back of the title page, then uncomment the
%% following line.
%\cleardoublepage

%% This generates the abstract page, with the line spacing adjusted
%% according to SGS guidelines.
\begin{abstract}
%% *** Put your Abstract here. ***
%% (At most 150 words for M.Sc. or 350 words for Ph.D.)
Social media such as Twitter have become an important method of communication, with potential opportunities for NLG to facilitate the generation of social media content. We focus on the generation of \emph{indicative tweets} that contain a link to an external web page. While it is natural and tempting to view the linked web page as the source text from which the tweet is generated in an extractive summarization setting, it is unclear to what extent actual indicative tweets behave like extractive summaries. We collect a corpus of indicative tweets with their associated articles and investigate to what extent they can be derived from the articles using extractive methods. We also consider the impact of the formality and genre of the article.

With the aim of finding the possible factors that influence the probability of the tweet being a summarization of the artivle, we conduct further studies to detect the function of the tweet. The results show that no particular intention correlates with the extraction of the tweets, but the tweets do seem to be summaries of the contents of the articles a majority of times. 

In the analysis of these questions, we have also generated a dataset of various tweets categorized by their subjects and  spanning broad areas of public interest, alongwith text in newspaper articles linked to by these tweets. Using the judgements provided by the human evaluators in our user study, we have also generated a human-tagged dataset.

Our results demonstrate the limits of viewing indicative tweet generation as extractive summarization, and point to the need for the development of a methodology for tweet generation that is sensitive to genre-specific issues.

\end{abstract}

\begin{abstract}
Les r\acute{e}seaux  sociaux tels que Twitter sont devenus un moyen de communication important, soulignant de plus en plus de possibilit\acute{e}s de g\acute{e}n\acute{e}ration automatique de texte (NLG) pour faciliter la g\acute{e}n\acute{e}ration des textes qu’on rencontre dans les r\acute{e}seaux  sociaux. Nous nous concentrons sur la g\acute{e}n\acute{e}ration de \ emph {tweets indicatifs} qui contiennent un lien pointant vers une page web externe. Bien qu’il soit naturel et tentant de consid\acute{e}rer que la page web li\acute{e}e soit la source à partir de laquelle le tweet est g\acute{e}n\acute{e}r\acute{e} dans un cadre de r\acute{e}sum\acute{e} automatique de texte par extraction, on ne sait pas dans quelle mesure les tweets indicatifs r\acute{e}els se comportent comme \acute{e}tant des r\acute{e}sum\acute{e}s par extraction. Nous collectons un corpus de tweets indicatifs avec leurs articles associ\acute{e}s et \acute{e}tudions dans quelle mesure ils peuvent être obtenus à partir des articles en utilisant des m\acute{e}thodes de r\acute{e}sum\acute{e} automatique par extraction. Nous consid\acute{e}rons \acute{e}galement l'effet de la formalit\acute{e} et le genre de l'article.

Dans le but de trouver les facteurs possibles qui agissent sur la probabilit\acute{e} du tweet \acute{e}tant un r\acute{e}sum\acute{e} de l'article, nous menons des \acute{e}tudes pour sp\acute{e}cifier et mieux comprendre la fonction du tweet. Les r\acute{e}sultats qui sont de port\acute{e}e consid\acute{e}rable d\acute{e}montrent une corr\acute{e}lation positive entre l ``extractivit\acute{e}" et l``informativit\acute{e}" du tweet. Dans l'analyse de ces questions, nous avons \acute{e}galement g\acute{e}n\acute{e}r\acute{e} un ensemble de donn\acute{e}es de diff\acute{e}rents tweets que nous avons class\acute{e} par leurs sujets et qui couvrent de grands domaines d'int\acute{e}rêt public, ainsi que le texte des articles de journaux auxquels pointent ces tweets. Les \acute{e}valuations fournies par les \acute{e}valuateurs humains dans notre \acute{e}tude des utilisateurs sont \acute{e}galement ajout\acute{e}es aux donn\acute{e}es pour produire un ensemble de donn\acute{e}es marqu\acute{e} par des individus pour identifier les fonctions des tweets.

Nos r\acute{e}sultats d\acute{e}montrent que considerer la g\acute{e}n\acute{e}ration des tweets indicatifs comme \acute{e}tant une m\acute{e}thode de r\acute{e}sum\acute{e} automatique par extraction a des limites. Ils soulignent \acute{e}galement qu’il est n\acute{e}cessaire de d\acute{e}velopper une m\acute{e}thodologie pour la g\acute{e}n\acute{e}ration des tweets qui est prend en consid\acute{e}ration les questions de genre.
\end{abstract}

% \chapter*{Acknowledgements}
\begin{acknowledgements}

I would like to thank my supervisor, Jackie Cheung, who has been a teacher and advisor and helped me not just by guiding my thesis, but also by creating a fun research group that I am proud to be a part of.

I would like to thank Prof. Joelle Pineau for putting me in touch with Jackie for my thesis, Jad Kabbara for helping me translate my abstract to French, and everyone in the very large Reasoning and Learning lab family.

I would also like to thank my parents, my family and friends who have been a constant emotional support system, to be always relied upon.

I would like to thank Julian Brooke for providing the formality lexicon used in part of this study, and Prof. Derek Ruths for agreeing to be the external reviewer of the thesis.

Last but not the least, I thank the McGill University community for a wonderful experience and for teaching me a thing or two beyond the subjects I came here to study.  
\end{acknowledgements}

%% Anything placed between the abstract and table of contents will
%% appear on a separate page since the abstract ends with \newpage and
%% the table of contents starts with \clearpage.  Use \cleardoublepage
%% for anything that you want to appear on a right-hand page.

%% This generates a "dedication" section, if needed
%% (uncomment to have it appear in the document).
%\begin{dedication}
%% *** Put your Dedication here. ***
%\end{dedication}

%% The `dedication' and `acknowledgements' sections do not create new
%% pages so if you want the two sections to appear on separate pages,
%% you should put an explicit \newpage between them.

%% This generates an "acknowledgements" section, if needed
%% (uncomment to have it appear in the document).
%\begin{acknowledgements}
%% *** Put your Acknowledgements here. ***
%\end{acknowledgements}

%% This generates the Table of Contents (on a separate page).
\tableofcontents

%% This generates the List of Tables (on a separate page), if needed
%% (uncomment to have it appear in the document).
\listoftables

%% This generates the List of Figures (on a separate page), if needed
%% (uncomment to have it appear in the document).
\listoffigures

%% You can add commands here to generate any other material that belongs
%% in the head matter (for example, List of Plates, Index of Symbols, or
%% List of Appendices).

%% End of the preliminary sections: reset page style and numbering.
\end{preliminary}


%%%%%%%%%%%%%%%%%%%%%%%%%%%%%%%%%%%%%%%%%%%%%%%%%%%%%%%%%%%%%%%%%%%%%%%%
%%  Put your Chapters here; the easiest way to do this is to keep     %%
%%  each chapter in a separate file and `\include' all the files.     %%
%%  Each chapter file should start with "\chapter{ChapterName}".      %%
%%  Note that using `\include' instead of `\input' will make each     %%
%%  chapter start on a new page, and allow you to format only parts   %%
%%  of your thesis at a time by using `\includeonly'.                 %%
%%%%%%%%%%%%%%%%%%%%%%%%%%%%%%%%%%%%%%%%%%%%%%%%%%%%%%%%%%%%%%%%%%%%%%%%

%% *** Include chapter files here. ***
\chapter{Introduction}
\label{chap:intro}

With the rise in popularity of social media, message broadcasting sites such as Twitter and other microblogging services have become an important means of communication, with an estimated 500 million tweets being written every day\footnote{https://about.twitter.com/company}. In addition to individual users, various organizations and public figures such as newspapers, government officials and entertainers have established themselves on social media in order to disseminate information or promote their products. 

\change{put later}While there has been recent progress in the development of Twitter-specific POS taggers, parsers, and other tweet understanding tools \cite{owoputi-etal-2013,kong-etal-2014}, there has been little work on methods for generating tweets, despite the utility this would have for users and organizations. 

In this thesis, we study the generation of the particular class of tweets that contain a link to an external web page that is composed primarily of text. Given the short length of a tweet, the presence of a URL in the tweet is a strong signal that the tweet is functioning to help Twitter users decide whether to read the full article. This class of tweets, which we call \emph{indicative tweets}, represents a large subset of tweets overall, constituting more than half of the tweets in our data set. Indicative tweets would appear to be the easiest to handle using current methods in text summarization, because there is a clear source of input from which a tweet could be generated. In effect, the tweet would be acting as an indicative summary of the article it is being linked to, and it would seem that existing methods in summarization can be applied. It should be noted that a tweet being indicative does not preclude it from also providing a critical evaluation of the linked article.

There has in fact been some work along these lines, within the framework of extractive summarization. An extractive summarization based tweet generator would find a great application in \cite{lofi2012iparticipate}, where they describe a system to generate tweets from local government records through keyphrase extraction. \cite{lloret2013towards} compares various extractive summarization algorithms applied on Twitter data to generate tweets from documents. 

Lofi and Krestel do not provide a formal evaluation of their model, while Lloret and Palomar compared overlap between system-generated and user-generated tweets using ROUGE 
\cite{lin2004rouge}. Unfortunately, they also show that there is little correlation between ROUGE scores and the perceived quality of the tweets when rated by human users for indicativeness and interest. More scrutiny is required to determine whether the wholesale adoption of methods and evaluation schemes from extractive summarization is justified.

Beyond issues of evaluation measures, it is also unclear whether extraction is the strategy employed by human tweeters. One of the original motivations behind extractive summarization was the observation that human summary writers tended to extract snippets of key phrases from the source text \cite{mani-2001}.  And while it may be true that an automatic tweet generation system need not necessarily follow the same approach to writing as human tweeters, it is still necessary to know what proportion of tweets could be accounted for in an extractive summarization paradigm. This paper does exactly this, and provides some explorative analysis and thus, some groundwork in developing a system that would be able to generate indicative tweets from referenced documents, possibly based on parameters for it to be specifically an advertisement or a public announcement and so on.

With indicative tweets, an additional issue arises in that the genre of the source text is not constrained; for example it may be a news article or an informal blog post or an advertisement. This may be vastly different from the desired formality of tweet itself, and thus, a genre-appropriate extract may not be available.

We begin to address the above issues through a study that examines to what extent tweet generation can be viewed as an extractive summarization problem. We extracted a dataset of indicative tweets containing a link to an external article, including the documents linked to by the tweets. We used this data and applied unigram, bigram and LCS (longest common subsequence) matching techniques inspired by ROUGE to determine what proportion of tweets can be found in the linked article. Even with the permissive unigram match measure, we find that well under half of the tweet can be found in the linked article. We also use stylistic analysis on the articles to examine the role that genre differences between the source text and the target tweet play, and find that it is easier to extract tweets from more formal articles than less formal ones.

We further conducted studies to identify functions, or the 'communicative intents' for the tweets with respect to the articles.
%add user study details

Our results point to the need for the development of a methodology for indicative tweet generation, rather than to expropriate the extractive summarization paradigm that was developed mostly on news text. Such a methodology will ideally be sensitive to stylistic factors as well as the underlying intent of the tweet.

\improvement{mention creation of dataset}

%outline of thesis chapters

The background and related work for this thesis is described in \chapref{chap:background}. The \chapref{chap:data} provides all the details on extraction and preprocessing of data, and the \chapref{chap:analysis} contains all the analysis done on this data. The function identification user studies is described in \chapref{chap:user}.  

\chapter{Background and Related Work}
\label{chap:background}

This chapter surveys various concepts used in the thesis, including an introduction to automatic summarization, methods for evaluating summaries, as well as studies relating to Twitter data and tweet generation.


\section{Summarization}

Text summarization is the task of condensing an original text document or documents while retaining as much of the important information as possible \citep{mani-2001}. In addition, the summary must also satisfy goals related to the quality of the generated text, such as readability and coherence. The two main approaches for automatic text summarization are \textit{extractive} and \textit{abstractive} summarization. 

Extractive summarization uses the technique of choosing the important parts of the text and rearranging them to generate summaries. \cite{nenkova2012survey} describe the components in extractive summarization techniques as 
\begin{inparaenum}[1)]
\item{building an internal representation of the important parts of the text,}
\item{ranking these in the order of importance or time or other relevant metric based on context, and then }
\item{selecting a suitable list of these sentences to form the summary.}
\end{inparaenum}

 This selection of important parts of the source can be done at various levels, such as phrases, sentences, or paragraphs \citep{nenkova2012survey, hahn2000challenges}. In phrase-level summarization, smoothing techniques may be used to generate readable texts, since stitching together phrases from the source will lack coherence. In contrast to phrase-level summarization, sentence-level summarization techniques tend to be inherently more grammatically correct since sentences are directly picked out. However, sentence compression techniques can also be used to reduce the size of the summaries in sentence-level summarization \citep{knight2002summarization}. Even after using smoothing techniques to generate readable text, extractive summaries tend to be incoherent and hard to read \citep{liu2009extractive}. It should be noted that since we are dealing with the summaries being used as tweets, and since tweets are only 140 characters long, we will mostly be dealing with word-level or n-gram-level summarization. 


% \begin{figure}[!htbp]
% \centering
% \includegraphics[width=\textwidth, height=4.5cm]{extractive}
% \caption{Example of extractive summarization.}
% \label{fig:extractive}
% \end{figure}


The second approach is that of abstractive summarization. This is a summarization approach that aims to keep the content or meaning of the input source the same while condensing the text or generalizing it, and involves text generation for generating the summary. As a rule, abstractive summarization requires world knowledge and is a much more difficult problem to solve. As a result, current summarization techniques concentrate on improving results from extractive summarization \citep{nenkova2012survey}. 

\begin{figure}[!t]
\centering
\includegraphics[width=\textwidth, height=4.5cm]{example_summarization}
\caption{Examples of extractive and abstractive summarization.}
\label{fig:examples}
\end{figure}

\figref{fig:examples} shows examples of both an extractive and abstractive summary in the form of a newspaper article thumbnail. The blue box outlines a few sentences from the article that have been picked to give a brief description. This acts as the extractive summary. The red box outlines the title of the article, which is an abstractive summary of the article; it is a generalization of the events described, carefully omitting details yet leaving the overall meaning of the event untouched.

% \figref{fig:extractive} shows an example of extractive summarization where a couple of sentences from the article have been picked to show in the the thumbnail for an article.

% \figref{fig:abstractive} shows the same thumbnail for a news article. However, the title of the article is an abstractive summary that is a generalization of the events described, carefully omitting details yet leaving the overall meaning of the event untouched.

% \begin{figure}[!htbp]
% \centering
% \includegraphics[width=\textwidth, height=4.5cm]{abstractive}
% \caption{Example of abstractive summarization.}
% \label{fig:abstractive}
% \end{figure}


Although extractive summarization has been predominant, there have been studies on its limitations. \cite{he2000comparing} compared user preferences for various mechanisms of browsing content from an audio-visual presentation. They demonstrated that the most preferred method of summarization was highlights and notes provided by the author, rather than transcripts or slides from the presentation, which can be viewed as the full source text and the compressed pointers for the presentation respectively. \cite{conroy2006topic} investigated the issue of limits of extraction by using an oracle ROUGE score based on a probabilistic model of unigrams that might appear in the gold standard summaries and exploit this to create a new method of summarization that uses maximum likelihood estimation.
% The oracle score is based on the maximum likelihood probability of words occurring in model summaries and is in turn used to generate summaries that perform better than any extracted and also human-generated summaries. 


\subsection{ROUGE: Evaluation Measure for Text Summarization}

ROUGE is an evaluation measure popularly used for evaluating the quality of summaries \citep{lin-2004}. It measures the quality of a summary by comparing the output of the system being tested against a set of gold standard summaries by word or n-gram overlap. The intuition is that if the generated summary has enough in common with a set of human-written summaries, then it can be judged as a good summary. The different types of comparisons calculated are the unigram, bigram, trigram and least common subsequence (ROUGE-1,2,3 and L respectively). A set of gold standard summaries are used to account for the fact that summary writers do not agree on the contents of the summary. 

\begin{equation}
\textit{ROUGE-n} = \frac{\sum\limits_{S \in \{Reference Summaries\}} \sum\limits_{gram_n \in S} Count_{match}(gram_n)}{\sum\limits_{S \in \{Reference Summaries\}} \sum\limits_{gram_n \in S} Count(gram_n)}
\end{equation}

The equation as described by \cite{lin2004looking} shows the calculation of the ROUGE-N score, where $n$ is the length of the n-gram, and $Count_{match}$ is the count of the matched n-grams in the summary and the reference summaries.

The use of multiple gold standard summaries gives rise to a subjective evaluation metric where the quality of the evaluation is dependent on the quality and number of the gold standard summaries. ROUGE also does not take into account whether the summary is fluent or coherent. However, ROUGE is useful for the evaluation of summarization methods for overall content retention from the original text. This is possible due to the use of n-gram co-occurrence statistics used by ROUGE.
% \section{Stylistics}
%\change{remove this and put it in the formality part}
% Stylistics is referred to the characteristics of text that can be extracted from it that do not relate to the meaning of the text. Common examples of these include textual statistics such as length of sentences and words, parts of speech, function words etc. and finds applications in authorship attribution, semantic analysis, personality typing and so on. Studies on building lexicons for formality have been conducted and are discussed further later \chapref{chap:analysis}.

\section{Studies Based on Twitter Data}

\subsection{Twitter Data and Summarization}

In this section, we discuss studies that use Twitter data as the source to study the summarization concepts discussed above.
% and other concepts which we consider in our user studies, such as classifying tweets based on who it was generated by, and sentiment analysis of tweets. 

The following studies focus on using summarization in relation to Twitter data. \cite{o2010tweetmotif} use topic summarization for a given search for better browsing. \cite{chakrabarti2011event} generate an event summary by learning about the event using a Hidden Markov Model over the tweets describing it. \cite{wang2014socially} generate a coherent event summary by treating summarization as an optimization problem for topic cohesion. \cite{inouye2011comparing} compare multiple summarization techniques to generate a summary of multi-post blogs on Twitter. \cite{wei2014utilizing} use tweets to help in generating better summaries of news articles.

% detailed analysis of this paper
As described in \chapref{chap:intro}, we analyze tweet generation using measures inspired by extractive summarization evaluation. \cite{lloret2013towards} compared the different text summarization techniques for tweet generation. Summarization systems were used to generate sentences which could then be taken to be tweets by summarizing documents to lengths smaller than 140 characters. The system-generated tweets were evaluated using ROUGE measures \citep{lin2004rouge}. The ROUGE-1, ROUGE-2 and ROUGE-L measures were used, and a human-written reference tweet was taken to be the gold standard. However, they found that the generated tweets did not rank well when evaluated by humans, even though the tweets achieved success in terms of ROUGE scores. %ROUGE has been known to work better when multiple reference summaries are used and is not meant to be used at the sentence level. This study uses ROUGE with a single reference summary, which is the reference tweet. However, given the size of a tweet, it can be argued that while generating a reference tweet from a single document, it is difficult to generate multiple reference tweets with largely varying content. \unsure{Is this reasoning okay?}

The \cite{lloret2013towards} study shows that extractive summarization algorithms may not generate good quality summaries despite giving high ROUGE evaluation scores. \cite{cheung2013towards} show that for the news genre, extractive summarization systems that are optimized for \textit{centrality}---that is, getting the core parts of the text into the summary---cannot perform well when compared to model summaries, since the model summaries are abstracted from the document to a large extent. Since ROUGE is such an evaluation method, we can call into question the use of extractive summarization for tweet generation in the news genre. This question fits into the bigger problem of whether extractive summarization can be used for tweet generation, and is discussed further in the following chapters.

\subsection{Classifying Twitter Data}
\label{sec:related_user}
%Having learnt from the tagging attempts earlier, we think that one promising direction is to model the \textit{function}, more explicitly. 

In this section we discuss concepts which we consider in our user studies, such as why the tweet was written, and what purpose does it serve.
% classifying tweets based on who it was generated by, and sentiment analysis of tweets. 

\cite{ghosh2011entropy} classified the retweeting activity of users based on time intervals between retweets of a single user and frequency of retweets from unique users. They defined `retweet' as the occurrence of the same URL in a different tweet. The study was able to classify the retweeting as automatic or robotic retweeting, campaigns, news, and blogs, based on the time-interval and user-frequency distributions. 

% In another study, \cite{chen2012extracting} were able to extract sentiment expressions from a corpus of tweets including both formal words and informal slang that bear sentiment.

We define \textit{function} of a tweet as why the user chose to write and share the tweet. This function can be considered on multiple levels. The closest description to our idea of function was found in \cite{sinclair1996preliminary}, who describe this as `communicative intent', describing the types as information, discussion, recommendation, recreation, religion and instruction with further subcategories. This higher level function is described as \textit{intent} in many studies, and is akin to classifying the topic or genre of the tweet. Examples of high-level function include indicative tweets, informative tweets or critical tweets. Lower-level functions of a tweet could be an advertisement for a car, an announcement of an event or drawing attention to a particular sentence in an article. 

There are several studies on classifying the function of tweets. \cite{wang2015mining} use bootstrapping to generate an intent keyword set used in generating an intent graph in a semi-supervised manner. They focus on finding tweets with intent and then classifying those tweets. The intent here is defined as a wish or a plan for some action, such as intent for buying/doing something such as food, drink, travel or career. Classification of intents in this way can directly be used as intents for purchasing and be utilized for advertisements. For example, if an intent for buying a new car is detected from a tweet by the user, advertisements of cars would be shown to the user. \citet{banerjee2012towards} analyze real time data to detect presence of intents in tweets. \citet{gomez2014content} use features from text and stylistics to determine user intentions, which are classified as news report, news opinion, publicity, general opinion, share location, chat, question or personal message. \cite{mohammad2013identifying} take a different approach on the intents and use them to study the classification of user intents specifically for tweets related to elections. They study tweets related to one election and classify tweets as ones that agree or disagree with the candidate,  or contain humour, support, sarcasm, or irony. They group these tweets into a broader classification of favouring vs opposing sentiments.
%These definitions of intent, while a promising start, will not be sufficient for tweet generation. For this purpose, intent would be the reason the user chose to share the article with that particular text. \cite{sinclair1996preliminary} describe this as 'communicative intent' describing the types as information, discussion, recommendation, recreation, religion and instruction with further subcategories. 


The studies discussed in this chapter lay the groundwork for the tasks performed in the next few chapters; namely, collecting the data, analyzing the data, and running user studies on the data. We now discuss the process of building the dataset in \chapref{chap:data}.


\chapter{Data Extraction and Preprocessing}
\label{chap:data}

This chapter discusses the process of data collection for the thesis including the need for a new dataset, the methods used for extracting data from Twitter and the preprocessing done on the data to get it ready for analysis. 


\section{Using Twitter for Data Extraction}

As mentioned earlier, there have been numerous studies that used data from the public Twitter feeds. However, none of the datasets in those studies focused on tweets and related articles linked to these tweets. The dataset of \cite{lloret2013towards} is an exception as it contains tweets and the news articles they link to. However, it only contains 200 English tweet-article pairs. \cite{wei2014utilizing} also constructed a dataset that contains both tweets and articles linked through them, but this data only deals with news text, and does not contain the variety of topics we wanted in the data. We therefore chose to build our own dataset. This section describes extraction, cleaning and other preprocessing of the data.

\section{Extracting Data}

Data was extracted from Twitter using the Twitter REST API using 51 search terms, or ‘hashtags’. These hashtags were chosen from a range of topics including pop culture,  international summit meetings discussing political issues, lawsuits and trials, social issues and health care issues. All these hashtags were ‘trending’ (being tweeted about at a high rate) at the time of extraction of the data. To get a broader sample, the data was extracted over the course of 15 days in November, 2014, which gave us multiple news stories to choose from for the search terms. The search terms were chosen so that there would be broad representation in terms of various stylistic properties of text like formality, subjectivity, etc. For example, searches related to politics would be more formal, while those related to films would be informal, and would also have a lot more opinion pieces about them. All the search terms used and their distribution in genre are shown in \tabref{tab:searchterms}.

We extracted about 30,000 tweets, of which more than half, or around 16,000, contained URLs to an external news article, photo on photo sharing sites, or video. The hashtags were chosen to maximise the number of articles related to the tweets. Many topics that were chosen were being tweeted about by news agencies and other popular news sources.


\begin{table}[!htbp]
\begin{tabular}{|l|l|l|l|}
\hline
Science \& Technology                                                                                                                               & Films \& Television                                                                                                                                                                              & Events                                                                                                                                                                         & Miscellaneous                                                                                                                                                                                                                                                 \\ \hline
\begin{tabular}[c]{@{}l@{}}\#rosetta\\ \#cometlanding\\ \#philae\\ \#lollipop\\ \#nexus6\\ \#lollipopupdate\\ \#android\\ \#mangalayan\end{tabular} & \begin{tabular}[c]{@{}l@{}}\#TaylorSwift\\ \#1989\\ \#theforceawakens\\ \#betterstarwarstitles\\ \#harrypotter\\ \#interstellar\\ \#johnoliver\\ \#BBCSyriaWars\\ \#montythepenguin\end{tabular} & \begin{tabular}[c]{@{}l@{}}\#haiyan\\ \#memorialday\\ \#winteriscoming\\ \#buffalosnow\\ \#snowstorm\\ \#nycmarathon\\ \#memorialday\\ \#lestweforget\\ \#bahamas\end{tabular} & \begin{tabular}[c]{@{}l@{}}\#beenrapedneverreported\\ \#1wtc\\ \#netneutrality\\ \#mentalhealth\\ \#MarysvilleShooting\\ \#ottawashootings\\ \#annefrank\\ \#obamacare\\ \#KevinVickers\\ \#RobertONeill\\ \#pointergate\\ \#abercrombieandfitch\end{tabular} \\ \hline
Politics                                                                                                                                            & International                                                                                                                                                                                    & Sports                                                                                                                                                                         & Trials                                                                                                                                                                                                                                                        \\ \hline
\begin{tabular}[c]{@{}l@{}}\#apec2014\\ \#apec\\ \#G20\\ \#GOP\\ \#cdnpoli\end{tabular}                                                             & \begin{tabular}[c]{@{}l@{}}\#berlinwall\\ \#ebola\\ \#erdogan\\ \#canadachinatradedeal\\ \#syria\\ \#putin\end{tabular}                                                                          & \begin{tabular}[c]{@{}l@{}}\#ausvssa\\ \#playingitmyway\\ \#nycmarathon\\ \#moneyball\end{tabular}                                                                             & \begin{tabular}[c]{@{}l@{}}\#oscarpistorius\\ \#ghomeshi\end{tabular}                                                                                                                                                                                         \\ \hline
\end{tabular}
\captionof{table}{Hashtags used for extraction, grouped into various categories.}
\label{tab:searchterms}
\end{table}

The data from the tweets was cleaned by removing the tweets that were not in English as well as the retweets; i.e., re-publications of a tweet by a different user.

We deduplicated the 16,000 extracted URLs into 6,003 unique addresses, then extracted and preprocessed their contents. The \texttt{newspaper} package\footnote{https://pypi.python.org/pypi/newspaper} was used to extract article text and the title from the web page. Since we are interested in text articles that can serve as the source text for summarization algorithms, we needed to remove photos and video links such as those from Instagram and YouTube. To do so, we removed those links that contained fewer than a threshold of 150 words. After this preprocessing, the number of useful articles was reduced from 6003 to 3066. There were some further tweet-article pairs where the text of the tweets was identical, these were removed by further preprocessing and the number of unique tweet-article pairs came down to 2471. 

The final version of the data consists of tweets along with other information about the tweet, such as links to articles, hashtags, time of publication, etc. We also retain the linked article text and preprocessed it using the CoreNLP toolkit developed by  \cite{manning2014stanford}. This includes the URL itself and the text extracted from the article, as well as some extracted information such as sentence boundaries, POS tags for tokens, parse trees and dependency trees. These annotations are used later during our analysis in \chapref{chap:analysis}. \tabref{tab:ex1} shows an example of an entry in the dataset.

\begin{table}[!htbp]
\centering
\begin{tabular}{|p{0.1\linewidth}|p{0.8\linewidth}|}
\hline
Tweet & \#RiggsReport: \#CA as the \#ElectionNight exception. Voters rewarded \#GOP nationally, but not in the \#GoldenState. http://t.co/K542wvSNVz \\ \hline
Title & The Riggs Report: California as the Election Night exception                                                                                 \\ \hline
Text  & When the dust settled on Election Night last week...                                                                                         \\ \hline
\end{tabular}
\captionof{table}{Example of a tweet, title of the article and the text.}
\label{tab:ex1}
\end{table}

A URL could have been tweeted through multiple tweets, all the ids of these tweets are linked to the same URL. It should be noted that the tweet to article dataset contains only the articles that are significantly long texts about the subject with a title, and contain no advertisements, other languages, or links to images or videos. 





\chapter{Analysis}
\label{chap:analysis}

We now describe the analyses we performed on the data in this chapter. The chapter begins by discussing the ways we arrived at the calculations made on the data, the details of how these tasks were performed using the data described in \chapref{chap:data} and then goes on the discuss the results and the consequent conclusions derived from the results.

Our goal is to investigate what proportion of the indicative tweets that we extracted can be found in the articles that they link to, in order to determine whether indicative tweet generation can be viewed as an extractive summarization problem. \tabref{tab:noextract} gives an example of data where the tweet that was shared about the article does not come directly from the article text, while \tabref{tab:extract} shows a tweet that was almost entirely extracted from the text of the article, but changed a little for the purpose of readability.

\begin{table}[!htbp]
\centering
\begin{tabular}{|p{0.1\linewidth}|p{0.8\linewidth}|}
\hline
Tweet &  Are \#Airlines doing enough with \#Ebola? http://t.co/XExWwxmjnk \#travel \\ \hline
Title &  Could shortsighted airline refund policies lead to an outbreak? \\  \hline
Text  &  The deadly Ebola virus has arrived in the United States just in time for the holiday travel season, carrying fear and uncertainty with it... \\ \hline
\end{tabular}
\captionof{table}{Example of a tweet, title of the article and the text when tweet cannot be extracted from text.}
\label{tab:noextract}
\end{table}

\begin{table}[!htbp]
\centering
\begin{tabular}{|p{0.1\linewidth}|p{0.8\linewidth}|}
\hline
Tweet & Officer \textbf{Wilson will be returned to active duty if no indictment}, says \#Ferguson Police \textbf{Chief} http://t.co/zrRIBxMUYJ  \\ \hline
Title & Jackson clarifies comments on Wilson's future status \\ \hline
Text  & ...\textbf{Chief} Jackson said if the grand jury does \textbf{not indict Wilson}, he \textbf{will} immediately \textbf{return to active duty}.... \\ \hline
\end{tabular}
\captionof{table}{Example of a tweet, title of the article and the text when tweet can be extracted from text. The matched portions of the tweet and article are in bold.}
\label{tab:extract}
\end{table}

We first compute the proportion of tweets that can be recovered directly from the article in its entirety (\secref{sec:exact-match}). Then, we calculate the degree of overlap in terms of unigrams and bigrams between the tweet and the text of the document (Sections \ref{sec:unigrams}, \ref{sec:bigrams}). 

In addition, we consider locality within the article when computing the overlap. For the unigram analysis, we performed a variant of the analysis, in which we computed the overlap within three-sentence windows in the source article (\secref{sec:window}). We also compute the least common subsequences between the tweet and the document (\secref{sec:lcs}). This was done to investigate whether sentence compression techniques could be applied to local context windows to generate the tweet.

These calculations are analogous to the ROUGE-1, -2 and -L style calculations. These results give an indication of the degree to which the tweet is extracted from the document text. 

For all these analyses, the stop words have been eliminated from the tweet as well as the document, so that only the informative words are taken into consideration. The comparisons were made without lemmatization or stemming, to adhere closely to existing work in extractive summarization, where the only modifications to the source text are removing discourse cue words or removing words by sentence compression techniques. The hashtags, references (@) and URLs from the tweets were all removed for analysis.

\section {Exact Match Calculations}
\label{sec:exact-match}
We first checked for a complete substring match of the tweet in the text. Out of the 2471 unique instances of tweet and article pairs, a complete match was found only 23 times. In 9 cases out of these, the tweet text matched the title of the article, which our preprocessing tool did not correctly separate from the body of the article. In the other cases, the text of the tweet appears in its entirety inside the body of the article. This suggests that the user chose the sentence that either seemed to be the most conclusive contribution of the article, or expressed the opinion of the user to be tweeted. An example for this is detailed in \tabref{tab:fullextract}.

Apart from the 9 times where the tweet was matched with title in the article, we also checked to see if the tweet text matched with the article titles that were separately extracted by the \texttt{newspaper} package in order to determine if tweets could be generated using the headline generation methods. We found that it did not match with the titles. However, even though there are no exact matches, there might still be matches where the tweet is a slight modification of the headline of the article, and can be measured using a partial match measure.

\begin{table}[!htbp]
\centering
\begin{tabular}{|p{0.1\linewidth}|p{0.8\linewidth}|}
\hline
Tweet &  @PNHP: \textbf{6. Renounce punitive and counterproductive measures such as “sealing the borders,”} http://t.co/LRLS2MhPRE \#Ebola \\ \hline
Title & Physicians for a National Health Program \\  \hline
Text  & As health professionals and trainees, we call on President Obama to take the following immediate steps to address the Ebola crisis... \textbf{6. Renounce punitive and counterproductive measures such as “sealing the borders,”} and take steps to address the... \\ \hline
\end{tabular}
\captionof{table}{Example where tweet is extracted as is from the text, matched portion in bold.}
\label{tab:fullextract}
\end{table}


\begin{figure}[!htbp]
\centering
\includegraphics[width=0.7\textwidth, height=9cm]{unigrammatch}
\caption[Unigram matching percentages]{Distribution of unigram match percentage over unique tweets and articles. The mean is 29.53\%, indicated by the red horizontal line, with a standard deviation of 20.2\%}
\label{fig:unigrammatch}
\end{figure}


\begin{figure}[!htbp]
\centering
\includegraphics[width=0.7\textwidth, height=9cm]{num_unigrams}
\caption[Histogram for number of unigrams matched]{Histogram of number of unique tweet-article pairs vs number of unigrams matched. The mean number of unigrams matched per tweet-article pair is 3.9.}
\label{fig:num_unigrams}
\end{figure}


\section{Percentage Match for Unigrams}
\label{sec:unigrams}

Next, we did a percentage match with the text of the article. This was a bag-of-words check using unigram overlap between the tweet and the document. Let $\textit{unigrams}(x)$ be the set of unigrams for some text $x$, then $u$, the percentage of matching unigrams found between a given tweet, $t$ and a given article, $a$, can be defined as  

\begin{equation}
u = \frac{| \textit{unigrams}(t) \cap \textit{unigrams}(a) |}{| \textit{unigrams}(t) |} * 100
\end{equation}

\figref{fig:unigrammatch} shows the percentage of matches in the tweet and the article text as compared to the number of unigrams in the tweet. The mean match percentage is 29.53\% and standard deviation is 20.2\%. The mean of this distribution shows that the number of matched unigrams from a tweet in the article are fairly low. \figref{fig:num_unigrams} shows the number of articles with a certain number of matching unigrams. The graph shows that the most common number of unigrams matched was 2. The number of articles with higher unigrams matched goes on decreasing. The slight rise at the end - more than 10 matched unigrams - is accounted for by the completely matched tweets described above.

\begin{figure}[!htbp]
\centering
\includegraphics[width=0.7\textwidth, height=9cm]{bigrammatch}
\caption[Bigram match percentages]{Distribution of bigram match percentage over the tweet-article pair. The mean here is 10.73\% shown by the red horizontal line, with a standard deviation of 18.5\%}
\label{fig:bigrammatch}
\end{figure}


\section{Percentage Match for Bigrams}
\label{sec:bigrams}

Similar to the unigram matching techniques, the bigram percentage matching was also calculated. The text of the tweet was converted into bigrams and we then looked for those bigrams in the article text. The percentage was calculated similar to the unigram matching done earlier. For the set of bigrams for a text $x$, $\textit{bigrams}(x)$, percentage of matching bigrams $b$ for the tweet $t$ and article $a$ is: 

\begin{equation}
b = \frac{| \textit{bigrams}(t) \cap \textit{bigrams}(a) |}{| \textit{bigrams}(t) |} * 100
\end{equation}

\figref{fig:bigrammatch} shows the percentages of matched bigrams found. The mean is 10.73 with a standard deviation of 18.5. As seen in the figure, most of the tweet-article pairs have no matched bigrams. The percentage then increases to reflect the complete matches found above.

\begin{figure}[!htbp]
\centering
\includegraphics[width=0.7\textwidth, height=9cm]{num_bigrams}
\caption[Histogram for number of matched bigrams.]{Histogram of number of unique tweet-article pairs vs number of bigrams matched. The mean number of bigrams matched per article is 1.9.}
\label{fig:num_bigrams}
\end{figure}

\figref{fig:num_bigrams} shows frequency of the number of tweet-article pairs for the number of bigrams matched. There are no matched bigrams for most of the pairs. A smaller number of articles had one matched bigram, and the number decreased until the end, where it increases a little at more than 10 matched bigrams because of exact tweet matches. 


\section{Percentage Match Inside a Window in the Article Text}
\label{sec:window}

The next analysis checks for a significant word matching inside a three-sentence window inside the article text. We used a three sentence long window using the sentence boundary information obtained during preprocessing. A window of three sentences was chosen to give a smaller context for the tweet to be extracted from than the entire article. The number was chosen as a moderate context window size as not too small to reduce it to a sentence level, and not too big for the context to be diluted. This was done to investigate whether a pseudo-extractive multi-sentence compression approach could convert a small number of sentences into a tweet.

After the text of the window was extracted, we performed a similar analysis as the last one, except on a smaller set of sentences. The matching percentages from all three-sentence windows in the articles were computed and the maximum out of these was taken for the final results. Let a sentence window $w_i$ be the set of three consecutive sentences starting from the sentence number $i$. For this window, the unigram match in the tweet $t$, and the window is the unigram match $u$ calculated above. Then, the maximum match from all the windows, $uw$ is 

\begin{equation}
uw = max_{w_i \in S} u(t, w_i)
\end{equation}

\begin{figure}[!htbp]
\centering
\includegraphics[width=0.7\textwidth, height=9cm]{unigramwindow}
\caption[Match percentages in tweet against window in article]{Percentages of common words in tweet and a three sentence window in the article. The maximum match from all percentages is chosen for an article. The red horizontal line is the mean is 26.6\%, and standard deviation 17\%.}
\label{fig:unigramwindow}
\end{figure}

The result from this experiment is shown in \figref{fig:unigramwindow}. Here, the mean of the values is 26.6\% and deviation 17\%. Again this shows that only a small proportion of tweets can be generated even with an approach that combines unigrams from multiple sentences in the article.


\section{Longest Common Subsequence Match Inside a Window for the Text}
\label{sec:lcs}

\begin{figure}[!htbp]
\centering
\includegraphics[width=0.7\textwidth, height=9cm]{lcs_doc}
\caption[match]{Percentages of words matching in tweet and document text using an LCS algorithm. Mean is 44.6\%, which is shown by the red horizontal line, and standard deviation is 22.7\%.}
\label{fig:lcs}
\end{figure}

The percentage match analyses were a bag-of-words approach that disregarded the order of the words inside the texts and tweets. To respect the order of the words in the sentence of the tweet, we also used the least common subsequence algorithm between the tweet text and the document text. This subsequence matching was done inside a sentence window of 5 sentences. 
Again, the final result for the article was the window in which the maximum percentage was recorded among all windows. The percentage match was calculated using the number of words in the tweet as the denominator.

If $\textit{lcs(t, a)}$ is the longest common subsequence between the tweet $t$ and article $a$, $\textit{unigrams}(x)$ is the set of unigrams for a text $x$, then the percentage of match for the lcs as compared to the tweet, $\textit{l}$ is


\begin{equation}
l = \frac{| \textit{lcs}(t, a) |}{| \textit{unigrams}(t) |} * 100
\end{equation}


 These numbers are shown in \figref{fig:lcs}. The mean here is 44.6\% and the standard deviation is 22.7\%. 

\section{Interaction with Formality}

As seen in the results of the analyses performed in \chapref{chap:analysis}, the tweets have little in common with the articles they are linked to. This shows that extractive summarization algorithms can only recover a small proportion of the indicative tweets. To tie in the results of the findings above with some intuitive notions about the text and see how formality interacts with the results, we also calculated the formality of the articles. This formality score was correlated with the longest common subsequence measure that we defined above. 

We assume that the formality of an article can be estimated by the formality of the words and phrases in the article. We used the formality lexicon of \cite{brooke2013multi}. They calculate formality scores for words and sentences by training a model on a large corpus based on the appearance of words in specific documents. Their model represents words as vectors and the formal and informal seeds appear in opposite halves of the graphs, suggesting that we can use these seeds to determine if an article is formal or informal. The lexicon consists of words and phrases and their degree of formality. Thus, more formal words are marked on a positive scale and informal words like those occurring in colloquial language are marked on a negative scale. 

Let the set of formality expressions from the lexicon be $L$, and the formality score for an expression $e$ be $\textit{score}(e)$. Let the set of all substrings from the article $\textit{substrings}(a)$ be $S$. Then, the formality score $f$ for an article $a$ is the number of formal expressions per 10 words in article is   

\begin{equation}
f = \frac{\sum\limits_{e \in L \& e \in S} \textit{score}(e)}{| \textit{unigrams}(a) |} * 10
\end{equation}

The formality lexicon gave positive weights for formal expressions and negative for informal expressions. When we computed $f$ using both formal and informal expressions, we found that the informal words predominated and ``swamped'' the signal of the formal words, leading to incomprehensible results. Thus, we discarded the informal words and used only the weights from the formal words in our final calculations. To check that these formality scores made sense intuitively, we calculated the average formality score for the articles belonging to each hashtag and ordered them, as shown in \tabref{tab:formal}.

\begin{table}[!htbp]
\centering
\begin{tabular}{|l|l|}
\hline
Lowest  & Highest  \\ \hline
\#theforceawakens       & \#KevinVickers           \\
\#TaylorSwift           & \#erdogan                \\
\#winteriscoming        & \#apec                  \\ \hline
\end{tabular}
\caption[Order of formality ranking in hashtags]{Table of hashtags (broadly, topics) with highest and lowest formality according to the lexicon.}
\label{tab:formal}
\end{table}


\begin{table}[!htbp]
\centering
\begin{tabular}{|p{0.1\linewidth}|p{0.8\linewidth}|}
\hline
Tweet &  @globetoronto: Why Buffalo got clobbered with snow and Toronto did not. \#weather \#snowstorm http://t.co/gcwwoDPZmX... http://t.co/BXY7EH6F3u" \\ \hline
Title & What caused Buffalo’s massive snow and why Toronto got lucky \\  \hline
Text  & Torontonians have long been the butt of jokes about calling in the army every time a few snow flurries whip by... \\ \hline
\end{tabular}
\caption[Example of formality in article affecting tweet]{Example of a tweet, title of the article where the formality of the article is over the mean, and the tweet is extracted from the article.}
\label{tab:formalexample}
\end{table}

This formality score for each article was correlated with the percentage of match obtained using the longest common subsequence algorithm. The Pearson correlation value was 0.41, with a p-value of 7.08e-66, indicating that the interaction between formality and overlap was highly significant. Hence, we can say that the more formal the subject or the article, the better the tweet can be extracted from the article. \tabref{tab:formalexample} gives an example of the formality of the article, which is a low 4.2 formality words per 10 words, where the tweet is not extracted from the article, but rephrased from the article instead.

We speculate that tweets associated with less formal articles may contain more abbreviations and non-standard words or spellings, which decreases the amount of overlap. To counter this case, we tried experimenting with word normalization systems which could resolve words like '2mw' or '4eva' to 'tomorrow' and 'forever' respectively. Systems described by \cite{yang2013log} and \cite{gouws2011contextual} were tested with our data. Unfortunately, neither provided the results we suitable enough to integrate with our analyses, and this will be something to look into further upon development of more accurate word normalization systems. 




\chapter{User Study for Identifying Functions of Tweets}
\label{chap:user}

% More details on the paper.
\chapref{chap:analysis} described the analyses performed on the data and the resulting conclusion that only a small portion of indicative tweets can be recovered from the article they link to if viewed as an extractive summarization problem. This conclusion suggests that the next step should be to gain more knowledge about the relation between the tweet and the article, specifically why the user chose to write the tweet. This information would enable us to infer whether a tweet can actually be generated from summarizing the text in the article and if so, lead us to a strategy that would be the most appropriate for this task. We explore the task of collecting more information about the dataset in this chapter. We run a user study on Amazon Mechanical Turk, a crowdsourcing platform that enables researchers to put up huge samples of data along with surveys, quizzes, and simple tasks to be solved by people over the world. In this study, we asked the users whether a tweet is an advertisement or a summary of the article. We describe the process of running the user study and analysis of results obtained from the user study.

\section{Functions of Tweets}
\label{sec:funcs}
% Preliminary recommendations on text typology
One aspect of gaining more knowledge would be asking what the \textit{`function'} of the tweet is. While many possible definitions of `function' are possible, our working definition of the term, introduced in \chapref{chap:background}, is why the user chose to share the article with the particular text written in the tweet. If the function of the tweet is known, it might be possible to predict the appropriateness of extractive methods for generating tweets. For example, if the function of the tweet is to be an advertisement for a particular product and the text contains the description of the features of the product, then intuitively, the tweet would likely be an extractive summary of the text. On the other hand, if the tweet were an attention grabbing title to merely bring traffic to the text, it would be more likely to be a generalized title, than a summary of the actual text. To identify these functions, we first examined multiple previous studies aiming to classify functions of tweets, discussed in \secref{sec:related_user}. Based on these studies, we isolated the functions and finalized the questions to be asked about the tweets. The data was then given to human evaluators to annotate in a user study.


\section{User Study Design}

The following sections describe the details surrounding the design and execution of the user study. 

\subsection{Questions Used in the User Study}
\label{sec:qs}

The questions to be asked in the user study with respect to each tweet-article pair came from our definition of the function of a tweet, described in \secref{sec:funcs}. The final questions that were used are shown in \tabref{tab:mturkqs}.  

\begin{table}[!t]
\centering
\begin{tabular}{|p{0.15\linewidth}|p{0.8\linewidth}|}
\hline
\textbf{Question 1:} Advertisement & Does the tweet explicitly encourage the reader to visit the link and read the original article? \\ \hline
\textbf{Question 2:} Summary & Does the tweet contain some information from the article, or summarize the article?             \\ \hline
\end{tabular}
\caption{Questions used in user study}
\label{tab:mturkqs}
\end{table} 

The type of articles being referenced, and the possible reasons these might be shared gave the list of possible functions. Using these and earlier studies, we suggest a list of relevant functions of tweets for our data: promote a product or an article or convey information from the article. These functions will ideally help provide parameters for generating tweets. The idea behind these functions will be discussed in the following paragraphs. 

\paragraph{Advertisement question} If the tweet references a newspaper article, it might be promoting the article, in the sense of attracting people to read the article in detail. This kind of tweet would try to sensationalize the material. It could either tweet the headline of the article directly, or summarize the headline itself further, or simply say something to the effect of `Check this out' or `This is worth a read' and then further tag the article with the use of appropriate hashtags indicating the contents of the text.

Example: ``Check out this article! \{url\}" or ``Look what this says! \{url\}"

\paragraph{Summary question} Secondly, the tweet could single out a particular piece of information or opinion directly from the article, either to agree with it or to express the importance of the sentence or phrase in the article according to the author of the tweet. It could also be a short summary of the text of the article either with the aim of inviting readers or just to inform readers about the contents of the article.

Example: ``Winter approaching, ways to stay safe from flu season: \{url\}"

These questions were presented separately in two different studies. We found a possibility that workers were viewing the questions in the pilot studies as either/or questions, where the answer to only one of them could be `yes'. Thus, separating the questions guaranteed an unbiased opinion about each question without any assumptions.

A third possible question of whether the tweet expressed an emotion towards the article, and if so, whether it was a positive or negative emotion was also considered. However, it was not included in the final study, and will be discussed in \secref{sec:pilot}

% \section{Running User Study on Amazon Mechanical Turk}

% Amazon Mechanical Turk is a crowdsourcing platform that enables researchers to put up huge samples of data along with surveys, quizzes, and simple tasks to be solved by people over the world.

%  \paragraph{Questions used} We devised three questions based on the possible functions described above, and different phrasing for these was experimented with. Each question asked if the tweet served that particular function with respect to the article. The question about whether the tweet conveyed any emotion about the article was dropped going forward with the rest of the studies because of the difficulty in distinguishing the difference between emotion towards the article vs emotion expressed in the article itself.


\subsection{Running the User Study}

\figref{fig:q1} and \figref{fig:q2} show an example of what a HIT (task on Mechanical Turk) for the two questions looked like to the user, respectively. For each of the questions, the first part of the figure shows the instructions involved as well as examples for every possible answer to the question asked. The tweet, the title that was extracted and the entire text of the article was then presented to help the workers make a decision about their answer.  

\begin{figure}[!htbp]
  \centering 
  \subfloat{\includegraphics[width=\textwidth, height=21cm]{q11}}
  %\qquad 
  %\subfloat[][b]{q11} 
  \caption[User study question 1 example]{(a) The first question (Advertisement) posed for each sample asked to the users.}
  \label{fig:q1}
\end{figure}

\begin{figure}[!htbp]
  \ContinuedFloat 
  \centering 
  \subfloat{\includegraphics[width=\textwidth, height=13cm]{q12}}% 
  %\qquad 
  %\subfloat[][]{q12} 
  \caption[]{(b) The first question (Advertisement) posed for each sample asked to the users.}
  \label{fig:q12}
\end{figure} 

\begin{figure}[!htbp]
  \centering 
  \subfloat{\includegraphics[width=\textwidth, height=21cm]{q21}}
  %\qquad 
  %\subfloat[][b]{q11} 
  \caption[User study question 2 example]{(a) The second question (Summary) asked for each sample.}
  \label{fig:q2}
\end{figure}

\begin{figure}[!htbp]
  \ContinuedFloat 
  \centering 
  \subfloat{\includegraphics[width=\textwidth, height=13cm]{q22}}% 
  %\qquad 
  %\subfloat[][]{q12} 
  \caption[]{(b) The second question (Summary) asked for each sample.}
  \label{fig:q22}
\end{figure} 

\subsection{Qualification Details}

Mechanical Turk assigns qualifications to workers based on their skill level in answering questions and the percentage of accepted answers. Workers with superior skills and higher accuracy while answering HITs are given the qualification `Master' by Mechanical Turk, which was the qualification used for these pilot studies.

Three separate opinions from the workers were gathered for each tweet and article pair and the inter-annotator agreement was calculated using Fleiss' kappa measure  of agreement \citep{geertzen2012inter} between the raters. 

We found that the way to obtain the best quality results was to add an additional level of qualification over the default Master's qualification. The qualification test contained three tweet-article pairs from the dataset that had an obvious category for function of tweets. The test was conducted separately for the two questions asked. These tests were presented to the workers, and only workers with a 100\% score were allowed to work on the data. These tests ensured that the worker had a complete understanding of the task at hand. A separate pilot study with these qualifications and questions structure gave promising results, and the user study was then run on the entire dataset.

\subsection{Pilot Studies}
\label{sec:pilot}
Before finalizing the design of the study, we considered various different options, which will be discussed further in this section.

\paragraph{Attempts at tagging data} Our first attempt at tagging the dataset was made using the following scheme: a sample set of 100 articles were tagged by two people, based on the title, the article and the tweet. The tags used in the preliminary tagging were `evaluative', `descriptive', and `mixed'. `Evaluative' text is a more opinionated text, while `Descriptive' text is non-evaluative, containing, for example, a narration of an event or an explanation about a certain object or event. A `mixed' category was also added during tagging to accommodate some articles that could fall in either category. It was observed that this classification was rather subjective. \cite{liu2012survey} give a detailed analysis on classifying evaluative vs descriptive texts, which is called as subjectivity classification. They point out that it is a bad idea to classify on a sentence level in complex sentences, since a sentence and by extension a text might be factual as well as evaluative, which was the original problem while tagging. This tagging scheme did not yield much success in terms of gleaning information over the entire dataset.

\paragraph{Pilot studies for the user study} Four pilot studies were conducted sequentially on 100 different randomly selected tweet and article pairs each time, using each to improve the design for the next study. For these pilot studies, we tried various combinations of questions and qualifications, which are a type of grading system based on experience and ability for workers on Mechanical Turk.
 
\paragraph{A third possible function of tweets} Another possible question that was briefly mentioned in \secref{sec:qs} is whether the tweet expresses an opinion about something in the article. It could be a positive or a negative comment about the contents of the article, or an agreement or disagreement about the events, thoughts, and opinions expressed in the article. 

Example: ``This is a terrible analysis. \{url\}" or ``Listening to this album on repeat! \{url\}"

% However, it was dropped after a couple of pilot studies because of the difficulty in separating the emotion expressed towards the article as opposed to simply reiterating the emotion expressed in the article itself.

However, this question was dropped going forward with the studies because of the difficulty in distinguishing the difference between emotion towards the article vs emotion reiterated from the article itself.


\section{Results and Analysis for the User Study}

This section describes the analysis of results obtained from the user study.
% \begin{figure}[!htbp]
% \centering
% \includegraphics[width=0.9\textwidth, height=9cm]{results}
% \caption[Results from user study]{Visualization for tags from human evaluators}
% \label{fig:res}
% \end{figure}

% The results obtained from the final study are shown in \figref{fig:res}. The bottom line shows the number of 'yes' votes per article and tweet for the first question, whether the tweet was a promotion of the article. The red line on top of it signifies the number of 'yes' votes for the second question, whether the tweet was a summary of the article. 

The total number of `yes' votes for each of the questions are shown in \tabref{tab:yeses}. \tabref{tab:exq1no}, \tabref{tab:exq1yes}, \tabref{tab:exq2no} and \tabref{tab:exq2yes} all show examples where for one of each question, all three workers agreed on their answers to each of the two questions. The Fleiss' kappa for the first study, showing the indicativeness of the tweet was 0.147 and the kappa for the second study, showing the informativeness of the tweet, was 0.208.



We then analyzed the answers obtained from the study by correlating them with the overlap measures defined in \chapref{chap:analysis} with the help of the Mann Whitney U test \citep{mann1947test,wilcoxon1947probability}. The Mann Whitney U test considers two groups of ratings and then analyses them in terms of rankings to infer how they corroborate. Each of the following tables shows a result for the Mann Whitney U test for a study pertaining to one of the questions, and a corresponding analysis: unigram, bigram, or LCS matching percentages. \tabref{tab:unicorr1} and \tabref{tab:unicorr2} show the test results for unigram match percentages, \tabref{tab:bicorr1} and \tabref{tab:bicorr2} for bigram match percentages, and \tabref{tab:lcscorr1} and \tabref{tab:lcscorr2} for longest common subsequence match percentages. 

Mechanical Turk presents each question for each tweet-article pair to three different workers. Thus, every tweet-article pair has three different opinions for each question asked. We split all the tweet-article pairs from the data into two different groups. We form the groups for the Mann-Whitney U test using the above information. The first split considers zero `yes' votes in the answers as one group and three `yes' votes as another group. The second split considers zero or  one `yes' votes out of three in one group and two or three `yes' votes out of three in the other group. For each of these studies for each sample set configuration, the U statistic and the p value are shown. The final two columns in both tables show the mean of values in each of the groups used in the test. 
%The failure to reject the null hypothesis suggests that no definite distinction can be made between the degree of extraction for tweets that are advertisements for articles or summaries of articles.


\begin{table}[!t]
\centering
\begin{tabular}{|l|l|l|l|l|}
\hline
\textbf{Questions}     & \textbf{0 `yes' votes} & \textbf{1 `yes' vote} & \textbf{2 `yes' votes} & \textbf{3 `yes' votes}\\ \hline
Q1: Advertisement & 53    & 340    & 942     &  1068  \\ \hline
Q2: Summary       & 29     & 176    & 703     & 1495    \\ \hline
\end{tabular}
\caption{Analysis of user study results}
\label{tab:yeses}
\end{table}


\begin{table}[!t]
\centering
\begin{tabular}{|p{0.1\linewidth}|p{0.8\linewidth}|}
\hline
\textbf{Tweet} &   RT @WSJ: In \#CometLanding Philae probe bounced and settled in area that could hinder its research. http://t.co/6lfg3p9XG1 http://t.co/A6fi  \\ \hline
\textbf{Title} &   Rosetta Mission Probe Landed on Comet in Shadow of Cliff	                                                                                 \\ \hline
\textbf{Text}  &  The historic Philae comet probe hit its target but then unexpectedly bounced twice settling in the shadow of a cliff that could hinder its research new images sent back Thursday showed.Philae is designed to run a suite...                                                                                         \\ \hline
\end{tabular}
\captionof{table}{Example where all three workers said it was not an advertisement.}
\label{tab:exq1no}
\end{table}


\begin{table}[!t]
\centering
\begin{tabular}{|p{0.1\linewidth}|p{0.8\linewidth}|}
\hline
\textbf{Tweet} & \#GalaxyNote3 \#Lollipop - SamMobile has been teasing us with a number of unfinished builds for a few http://t.co/A0IKYsk4g3 \#Samsung \\ \hline
\textbf{Title} &   Samsung GALAXY Note 3's Android Lollipop Update Surfaces                                                                                 \\ \hline
\textbf{Text}  &  SamMobile has been teasing us with a number of unfinished builds for a few months now. This indicates...                                                                                         \\ \hline
\end{tabular}
\captionof{table}{Example where all three workers said the tweet was an advertisement for article.}
\label{tab:exq1yes}
\end{table}


\begin{table}[!t]
\centering
\begin{tabular}{|p{0.1\linewidth}|p{0.8\linewidth}|}
\hline
\textbf{Tweet} & "RT @jakbarali: So my partner Gillian Hnatiw and I had something to say about \#VAW \#LoriDouglas and \#Ghomeshi. http://t.co/6X2zMtCAM0 \\ \hline
\textbf{Title} & "Victim-blaming couched as legitimate judicial inquiry" \\ \hline
\textbf{Text}  & Ghomeshi himself broke the first wave of the story when he took to Facebook to decry the CBCs decision to terminate him... \\ \hline
\end{tabular}
\captionof{table}{Example where all three raters said the tweet was not a summary.}
\label{tab:exq2no}
\end{table}

\begin{table}[!t]
\centering
\begin{tabular}{|p{0.1\linewidth}|p{0.8\linewidth}|}
\hline
\textbf{Tweet} & RT @PopCulturPriest: Doing a story on California's lottery for @americmag I discovered \#JohnOliver's story had some troubling errors: http \\ \hline
\textbf{Title} & Blowing The Dismount: Last Week Tonight Fudges Its Lottery Story \\ \hline
\textbf{Text}  & Sunday night on the season finale of HBOs new news show Last Week Tonight anchor John Oliver spent half the show... \\ \hline
\end{tabular}
\captionof{table}{Example where all three raters agreed the tweet was a summary.}
\label{tab:exq2yes}
\end{table}

\begin{table}[!htbp]
\centering
\begin{tabular}{|p{0.29\textwidth}|p{0.1\textwidth}|p{0.1\textwidth}|p{0.09\textwidth}|p{0.09\textwidth}|p{0.09\textwidth}|p{0.09\textwidth}|}
\hline
\textbf{Groups considered}    & \textbf{U statistic} & \textbf{p value} & \textbf{Mean of values for Group 1} & \textbf{Number of samples in Group 1} & \textbf{Mean of values for Group 2} & \textbf{Number of samples in Group 2} \\ \hline
Group 1: 0 `yes' votes \newline Group 2: 3 `yes' votes &  28104  &  0.931413  &  27.44  & 53 & 28.21 &   1068  \\ \hline
Group 1: 0 or 1 `yes' votes \newline Group 2: 2 or 3 `yes' votes &   406355.5  & 0.365219 & 30.65  & 393 & 29.23 & 2010 \\ \hline
\end{tabular}
\caption{Mann Whitney U test results for the advertisement question(indicativeness): Unigram Match}
\label{tab:unicorr1}
\end{table}

\begin{table}[!htbp]
\centering
\begin{tabular}{|p{0.29\textwidth}|p{0.1\textwidth}|p{0.1\textwidth}|p{0.09\textwidth}|p{0.09\textwidth}|p{0.09\textwidth}|p{0.09\textwidth}|}
\hline
\textbf{Groups considered}    & \textbf{U statistic} & \textbf{p value} & \textbf{Mean of values for Group 1} & \textbf{Number of samples in Group 1} & \textbf{Mean of values for Group 2} & \textbf{Number of samples in Group 2}\\ \hline
Group 1: 0 `yes' votes \newline Group 2: 3 `yes' votes & 12211 & 0.000055  & 16.69 & 29 & 31.07 & 1495  \\ \hline
Group 1: 0 or 1 `yes' votes \newline Group 2: 2 or 3 `yes' votes & 193411 & 0.000791 & 25.08 & 205 & 29.87 & 2198 \\ \hline
\end{tabular}
\caption{Mann Whitney U test results for the summary question(informativeness): Unigram Match}
\label{tab:unicorr2}
\end{table}

\begin{table}[!htbp]
\centering
\begin{tabular}{|p{0.29\textwidth}|p{0.1\textwidth}|p{0.1\textwidth}|p{0.09\textwidth}|p{0.09\textwidth}|p{0.09\textwidth}|p{0.09\textwidth}|}
\hline
\textbf{Groups considered}    & \textbf{U statistic} & \textbf{p value} & \textbf{Mean of values for Group 1} & \textbf{Number of samples in Group 1} & \textbf{Mean of values for Group 2} & \textbf{Number of samples in Group 2}\\ \hline
Group 1: 0 `yes' votes \newline Group 2: 3 `yes' votes & 27871 & 0.851388 & 8.31 & 53 & 9.21 & 1068  \\ \hline
Group 1: 0 or 1 `yes' votes \newline Group 2: 2 or 3 `yes' votes& 406313 & 0.378553 & 12.19 & 393 & 10.29 & 2009 \\ \hline
\end{tabular}
\caption{Mann Whitney U test results for the advertisement question(indicativeness): Bigram Match}
\label{tab:bicorr1}
\end{table}

\begin{table}[!htbp]
\centering
\begin{tabular}{|p{0.29\textwidth}|p{0.1\textwidth}|p{0.1\textwidth}|p{0.09\textwidth}|p{0.09\textwidth}|p{0.09\textwidth}|p{0.09\textwidth}|}
\hline
\textbf{Groups considered}    & \textbf{U statistic} & \textbf{p value} & \textbf{Mean of values for Group 1} & \textbf{Number of samples in Group 1} & \textbf{Mean of values for Group 2} & \textbf{Number of samples in Group 2}\\ \hline
Group 1: 0 `yes' votes \newline Group 2: 3 `yes' votes & 15006 & 0.004541 & 3.88 & 29 & 11.46 & 1494  \\ \hline
Group 1: 0 or 1 `yes' votes \newline Group 2: 2 or 3 `yes' votes & 201755.5 & 0.013592 & 8.07 & 205 & 10.84 & 2197 \\ \hline
\end{tabular}
\caption{Mann Whitney U test results for the summary question(informativeness): Bigram Match}
\label{tab:bicorr2}
\end{table}



\begin{table}[!htbp]
\centering
\begin{tabular}{|p{0.29\textwidth}|p{0.1\textwidth}|p{0.1\textwidth}|p{0.09\textwidth}|p{0.09\textwidth}|p{0.09\textwidth}|p{0.09\textwidth}|}
\hline
\textbf{Groups considered}    & \textbf{U statistic} & \textbf{p value} & \textbf{Mean of values for Group 1} & \textbf{Number of samples in Group 1} & \textbf{Mean of values for Group 2} & \textbf{Number of samples in Group 2}\\ \hline
Group 1: 0 `yes' votes \newline Group 2: 3 `yes' votes &  26440.5  &  0.418424  &  42.16  & 53 &  44.24 &   1068  \\ \hline
Group 1: 0 or 1 `yes' votes \newline Group 2: 2 or 3 `yes' votes  &  392910     & 0.870236 &  44.66  & 393 & 44.69 & 2010 \\ \hline
\end{tabular}
\caption{Mann Whitney U test results for the advertisement question(indicativeness): Longest Common Subsequence}
\label{tab:lcscorr1}
\end{table}

\begin{table}[!htbp]
\centering
% \setlength\extrarowheight{5pt}
\begin{tabular}{|p{0.29\textwidth}|p{0.1\textwidth}|p{0.1\textwidth}|p{0.09\textwidth}|p{0.09\textwidth}|p{0.09\textwidth}|p{0.09\textwidth}|}
\hline
\textbf{Groups considered}    & \textbf{U statistic} & \textbf{p value} & \textbf{Mean of values for Group 1} & \textbf{Number of samples in Group 1} & \textbf{Mean of values for Group 2} & \textbf{Number of samples in Group 2}\\ \hline
Group 1: 0 `yes' votes \newline Group 2: 3 `yes' votes & 18466   & 0.171255 & 38.4 & 29 & 44.47  & 1495 \\ \hline
Group 1: 0 or 1 `yes' votes \newline Group 2: 2 or 3 `yes' votes & 217196.5      &  0.393999    & 43.16    & 205 & 44.83   &  2198\\ \hline
\end{tabular}
\caption{Mann Whitney U test results for the summary question(informativeness): Longest Common Subsequence}
\label{tab:lcscorr2}
\end{table}

The p-values for \tabref{tab:unicorr1}, \tabref{tab:bicorr1} and \tabref{tab:lcscorr1} show non-significant results for both sets of groups for the first question, the indicativesness of the tweet. The U statistic for each case is very high and the results show a $p>0.05$. We thus fail to reject the null hypothesis that the two sets were pulled from the same distribution. For all these cases, the means of the two groups are very close to the means for the respective analyis, and to each other. Mean for Unigram match is 29.53\%, mean for bigram match is 10.73\% and the mean for LCS match is 44.6\% as seen in \chapref{chap:analysis}. 
%The U statistic for each case is very high and the results show a $p>0.5$. We thus fail to reject the null hypothesis, that the two sets were pulled from the same distribution. The means of the LCS values, indicating the extractiveness of the tweet, are very close to each other. 

The p-values for unigram and bigram match for the second question, indicating the informativeness of the summary, shown in \tabref{tab:unicorr2} and \tabref{tab:bicorr2} are both significant, with $p<0.05$, especially so for the first arrangement of groups where group 1 is zero `yes' votes and group 2 is three `yes' votes. Based on the result of the p-values, we can conclude that these samples are drawn from different populations. If we look at the means of the values in each case, they are sufficiently different, with the mean of the first group being significantly smaller than the mean of the values in the second group. \tabref{tab:lcscorr2} also shows a slight difference in the means when zero vs three `yes' votes were considered as the sample set configuration. The U-statistic and p-value are both the least in this case for longest common subsequence results. However, no significant result can be drawn from this since the p-value is still quite high. It is possible that the non-significant result can be explained by the fact that the LCS is a lot more flexible for accommodating words from the overall article, and thus while the means of the two groups show difference in the right direction, the p-value is still too	 high to conclude anything significant. 

\subsection{Conclusions from the User Study}

The significant results from \tabref{tab:unicorr2} and \tabref{tab:bicorr2} represent evidence that tweets that are informative and tweets that are not informative have different levels of extractiveness from their source article. However, the evidence does not support the fact that whether a tweet is an advertisement interacts with the extractiveness of the tweet. Further studies would be required to come to a conclusion about this type of summary classification based on function, and how it interacts with extractiveness of the summary. The study shows a promising direction for further studies on the function of tweets. 

An important outcome of this chapter is the generation of a human-tagged dataset of tweet and article pairs, based on the indicativeness and informativeness of the tweets with respect to the article text. 

The question of whether a tweet summarizes the content of the article gave mostly positive answers, suggesting that according to the workers, if the tweet contained a link to article, it was an indicative summary in most cases. However, according to the extractiveness calculated earlier in \chapref{chap:analysis}, the tweets were not extracted from the articles to a large extent. With the results from the user study performed in this chapter, we can see that even when the tweet is used informatively, extractive methods have an upper bound that is still low, similar to what was obtained earlier. This reinforces the earlier conclusion of a need for a more sophisticated tool that summarizes the contents of the article for tweet generation.

% The question about whether a tweet summarizes the content of the article gave mostly postive answers, and the correlation with extractiveness shows that is the tweet is used informatively, then extractive methods give us a higher bound which still not helpful for generating tweets. This reinforces the earlier assumption that tweet generation must be done using a more sophisticated tool that summarizes the contents of the article.

%http://www.graphpad.com/guides/prism/6/statistics/index.htm?how_the_mann-whitney_test_works.htm




\chapter{Conclusion}
\label{chap:conclusion}

We have described a study that investigates whether indicative tweet generation can be viewed as an extractive summarization problem. By analyzing a dataset of indicative tweets that we collected using measures inspired by extractive summarization evaluation, we find that most tweets cannot be recovered from the article that they link to, demonstrating a limit to the effectiveness of extractive methods.

We further performed an analysis to determine the role of formality differences between the source article and the Twitter genre. We find evidence that formality is an important factor, as the less formal the source article is, the less extractive the tweets seem to be. Future methods that can change the level of formality of a piece of text without changing the contents will be needed, as will those that explicitly consider the intended use of the tweet.

Finally, we conducted a study to determine whether the function of the tweet towards the article was a factor in the degree to which the tweet was extracted from the article. The  analyses performed in \chapref{chap:analysis} show that a small percentage of tweets can be extracted from articles. The user study further confirms that a majority of articles are summaries of the articles, according to the workers. This shows that it is worth pursuing abstractive summarization as a way to generate tweets. We have consequently generated a dataset of tweets and articles categorized by topic, and asked users to tag them according to whether the tweet is an advertisement encouraging the user to click on and read the entire article, or a summary of the article. This generated dataset of tagged tweets and articles is an important contribution of the thesis, and can be used in further studies towards identifying functions of tweets and also in tweet generation.

\section{Future Work}

\subsection{Study Functions and Intents}
Our studies of communicative functions have explored two aspects of the functions of tweets. It would be worthwhile to further explore the reasons for writing tweets, to be able to classify them, and use this information further as parameters for advertisements or personalized feeds. Analysis of the text and the tweet itself in conjunction with the various intents described in \cite{sinclair1996preliminary} would help to solve the problem.

\subsection{A Structure for Generating Tweets}
The final goal would be the ability to generate a tweet based on the text of the article or a blog, possibly with the help of a parameter: a communicative goal mentioned above. The communicative goal would help establish the context in which the tweet would be used and therefore the kind of tweet that needs to be generated from the text.  

\subsection{Parameterized summarization}
A broader parameterized text summarization system would be an excellent generalization of the tweet generation process. This would not only include a way to generate a summary according to the way in which the summary would be used, but also consider what the summary intends to convey from the text. For example, a summary could be converted to a higher or a lower level of formality for publishing to different outlets. A summary posted on a social media platform would be less formal whereas a summary posted on a blog would be comparatively more formal.


%% This adds a line for the Bibliography in the Table of Contents.
\addcontentsline{toc}{chapter}{Bibliography}
%% *** Set the bibliography style. ***
%% (change according to your preference/requirements)
\bibliographystyle{unsrtnat}
%% *** Set the bibliography file. ***
%% ("thesis.bib" by default; change as needed)
\bibliography{thesis}

%% *** NOTE ***
%% If you don't use bibliography files, comment out the previous line
%% and use \begin{thebibliography}...\end{thebibliography}.  (In that
%% case, you should probably put the bibliography in a separate file and
%% `\include' or `\input' it here).

\end{document}
