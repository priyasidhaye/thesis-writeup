\chapter{Conclusion}
\label{chap:conclusion}

We have described a study that investigates whether indicative tweet generation can be viewed as an extractive summarization problem. By analyzing a dataset of indicative tweets that we collected using measures inspired by extractive summarization evaluation, we find that most tweets cannot be recovered from the article that they link to, demonstrating a limit to the effectiveness of extractive methods.

We further performed an analysis to determine the role of formality differences between the source article and the Twitter genre. We find evidence that formality is an important factor, as the less formal the source article is, the less extractive the tweets seem to be. Future methods that can change the level of formality of a piece of text without changing the contents will be needed, as will those that explicitly consider the intended use of the tweet.

Finally, we conducted a study to determine whether the function of the tweet towards the article was a factor in the degree to which the tweet was extracted from the article. The  analyses performed in \chapref{chap:analysis} show that a small percentage of tweets can be extracted from articles. The user study further confirms that a majority of articles are summaries of the articles, according to the workers. This shows that it is worth pursuing abstractive summarization as a way to generate tweets. We have consequently generated a dataset of tweets and articles categorized by topic, and asked users to tag them according to whether the tweet is an advertisement encouraging the user to click on and read the entire article, or a summary of the article. This generated dataset of tagged tweets and articles is an important contribution of the thesis, and can be used in further studies towards identifying functions of tweets and also in tweet generation.

\section{Future Work}

\subsection{Study Functions and Intents}
Our studies of communicative functions have explored two aspects of the functions of tweets. It would be worthwhile to further explore the reasons for writing tweets, to be able to classify them, and use this information further as parameters for advertisements or personalized feeds. Analysis of the text and the tweet itself in conjunction with the various intents described in \cite{sinclair1996preliminary} would help to solve the problem.

\subsection{A Structure for Generating Tweets}
The final goal would be the ability to generate a tweet based on the text of the article or a blog, possibly with the help of a parameter: a communicative goal mentioned above. The communicative goal would help establish the context in which the tweet would be used and therefore the kind of tweet that needs to be generated from the text.  

\subsection{Parameterized summarization}
A broader parameterized text summarization system would be an excellent generalization of the tweet generation process. This would not only include a way to generate a summary according to the way in which the summary would be used, but also consider what the summary intends to convey from the text. For example, a summary could be converted to a higher or a lower level of formality for publishing to different outlets. A summary posted on a social media platform would be less formal whereas a summary posted on a blog would be comparatively more formal.