\chapter{Conclusion}
We have described a study that investigates whether indicative tweet generation can be viewed as an extractive summarization problem. By analyzing a collection of indicative tweets that we collected according to measures inspired by extractive summarization evaluation measures, we find that most tweets cannot be recovered from the article that they link to, demonstrating a limit to the effectiveness of extractive methods.

We further performed an analysis to determine the role of formality differences between the source article and the Twitter genre. We find evidence that formality is an important factor, as the less formal the source article is, the less extractive the tweets seem to be. Future methods that can change the level of formality of a piece of text without changing the contents will be needed, as will those that explicitly consider the intended use of the tweet.

Finally, we conducted a study to determine whether the function of the tweet towards the article, or the function was a factor in the degree to which the tweet was extracted from the article. The fact that the earlier analyses show a small number of extracted tweets, but user study results showing the maximum number of articles being summaries of the articles validates the previous claim that an abstraction based summarization system would be necessary to generate tweets. We have subsequently generated a data set of tweets and articles categorized by topics, with tags about whether the tweet is an advertisement encouraging the user to click on and take a further look at the article, a summary of the article. This also includes if the tweet has something positive/negative to say about the article text itself, rather than the subject of the article. This generated dataset of tagged tweets and articles is an important contribution of the thesis, and can be used in further studies towards identifying functions of tweets and also in tweet generation.

\section{Future Work}

\subsection{Study functions and intents}
The studies looking at the communicative intents have explored one aspect of the function of the tweet. It would be worthwhile to look more into the reasons for writing tweets, to be able to classify them and use this information further as parameters for advertisements, personalized feeds and so on. Analysis of the text and the tweet itself in conjunction with the various intents described in \cite{sinclair1996preliminary} would help solve the problem.

\subsection{A structure for generating tweets}
The final goal would be the ability to generate a tweet based on the text of the article or a blog, possibly with the help of a parameter : a communicative goal mentioned above. The communicative goal would help establish the context in which the tweet would be used and therefore the kind of tweet that needs to be generated from the text.  

\subsection{Parametrized summarization}
A broader parametrized text summarization system would be an excellent generalization of the tweet generation process. This would not only include a way to generate a summary as per the way in which the summary would be used, but also consider what is intended for the summary to convey from the text. For example, for a summary to convey a positive impression of a restaurant from a detailed mixed review, it would be choosing the positive parts of the text while masking over the negative parts to generate a parameter-based summary.