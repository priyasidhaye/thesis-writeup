Les r\acute{e}seaux  sociaux tels que Twitter sont devenus un moyen de communication important, soulignant de plus en plus de possibilit\acute{e}s de g\acute{e}n\acute{e}ration automatique de texte (NLG) pour faciliter la g\acute{e}n\acute{e}ration des textes qu’on rencontre dans les r\acute{e}seaux  sociaux. Nous nous concentrons sur la g\acute{e}n\acute{e}ration de \ emph {tweets indicatifs} qui contiennent un lien pointant vers une page web externe. Bien qu’il soit naturel et tentant de consid\acute{e}rer que la page web li\acute{e}e soit la source à partir de laquelle le tweet est g\acute{e}n\acute{e}r\acute{e} dans un cadre de r\acute{e}sum\acute{e} automatique de texte par extraction, on ne sait pas dans quelle mesure les tweets indicatifs r\acute{e}els se comportent comme \acute{e}tant des r\acute{e}sum\acute{e}s par extraction. Nous collectons un corpus de tweets indicatifs avec leurs articles associ\acute{e}s et \acute{e}tudions dans quelle mesure ils peuvent être obtenus à partir des articles en utilisant des m\acute{e}thodes de r\acute{e}sum\acute{e} automatique par extraction. Nous consid\acute{e}rons \acute{e}galement l'effet de la formalit\acute{e} et le genre de l'article.

Dans le but de trouver les facteurs possibles qui agissent sur la probabilit\acute{e} du tweet \acute{e}tant un r\acute{e}sum\acute{e} de l'article, nous menons des \acute{e}tudes pour sp\acute{e}cifier et mieux comprendre la fonction du tweet. Les r\acute{e}sultats qui sont de port\acute{e}e consid\acute{e}rable d\acute{e}montrent une corr\acute{e}lation positive entre l ``extractivit\acute{e}" et l``informativit\acute{e}" du tweet. Dans l'analyse de ces questions, nous avons \acute{e}galement g\acute{e}n\acute{e}r\acute{e} un ensemble de donn\acute{e}es de diff\acute{e}rents tweets que nous avons class\acute{e} par leurs sujets et qui couvrent de grands domaines d'int\acute{e}rêt public, ainsi que le texte des articles de journaux auxquels pointent ces tweets. Les \acute{e}valuations fournies par les \acute{e}valuateurs humains dans notre \acute{e}tude des utilisateurs sont \acute{e}galement ajout\acute{e}es aux donn\acute{e}es pour produire un ensemble de donn\acute{e}es marqu\acute{e} par des individus pour identifier les fonctions des tweets.

Nos r\acute{e}sultats d\acute{e}montrent que considerer la g\acute{e}n\acute{e}ration des tweets indicatifs comme \acute{e}tant une m\acute{e}thode de r\acute{e}sum\acute{e} automatique par extraction a des limites. Ils soulignent \acute{e}galement qu’il est n\acute{e}cessaire de d\acute{e}velopper une m\acute{e}thodologie pour la g\acute{e}n\acute{e}ration des tweets qui est prend en consid\acute{e}ration les questions de genre.