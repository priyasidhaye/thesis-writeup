\chapter{Introduction}
\label{chap:intro}

Social media comprise a large part of our lives right now, with various outlets providing a platform for sharing thoughts, news, pictures and so on. With the rise in popularity of social media, message broadcasting sites such as Twitter and other microblogging services have become an important means of communication, with an estimated 500 million tweets being written every day\footnote{https://about.twitter.com/company}. In addition to individual users, various organizations and public figures such as newspapers, government officials and entertainers have established themselves on social media in order to disseminate information or promote their products. Social media thus provide an incredibly dense and varied source of data, originating from people all over the world, voicing opinions and sharing news around them very easily and consistently.

Naturally, there has been an increase in the number of studies using the sources of data social media provides. Specifically for Twitter, these include areas such as tweet parsing, text normalization, tweet POS tagging, classification of tweets, event summarization, identifying bot behaviours and inferring things like political views of individuals from this text. While this progress in the development of Twitter-specific POS taggers, parsers, and other tweet understanding tools \cite{owoputi-etal-2013,kong-etal-2014} is going strongly, there has been little work on methods for generating tweets, despite the utility this would have for users and organizations, for purposes of advertisement, education, or even entertainment. 

In this thesis, we study the generation of the particular class of tweets that contain a link to an external web page that is composed primarily of text. Given the short length of a tweet, the presence of a URL in the tweet is a strong signal that the tweet is functioning to help Twitter users decide whether to read the full article. This class of tweets, which we call \emph{indicative tweets}, represents a large subset of tweets overall, constituting more than half of the tweets in our data set. Indicative tweets would appear to be the easiest to handle using current methods in text summarization, because there is a clear source of input from which a tweet could be generated. In effect, the tweet would be acting as an indicative summary of the article it is being linked to, and it would seem that existing methods in summarization can be applied. It should be noted that a tweet being indicative does not preclude it from also providing a critical evaluation of the linked article.

There has in fact been some work along these lines, within the framework of extractive summarization. An extractive summarization based tweet generator would find a great application in \cite{lofi2012iparticipate}, where they describe a system to generate tweets from local government records through keyphrase extraction. \cite{lloret2013towards} compares various extractive summarization algorithms applied on Twitter data to generate tweets from documents. 

Lofi and Krestel do not provide a formal evaluation of their model, while Lloret and Palomar compared overlap between system-generated and user-generated tweets using ROUGE, 
\citep{lin2004rouge}, a recall-based evaluation metric for summarization. Unfortunately, they also show that there is little correlation between ROUGE scores and the perceived quality of the tweets when rated by human users for indicativeness and interest. More scrutiny is required to determine whether the wholesale adoption of methods and evaluation schemes from extractive summarization is justified. More studies are discussed in \chapref{chap:background}.

Beyond issues of evaluation measures, it is also unclear whether extraction is the strategy employed by human tweeters. One of the original motivations behind extractive summarization was the observation that human summary writers tended to extract snippets of key phrases from the source text \cite{mani-2001}.  And while it may be true that an automatic tweet generation system need not necessarily follow the same approach to writing as human tweeters, it is still necessary to know what proportion of tweets could be accounted for in an extractive summarization paradigm. This thesis does exactly this, and provides some explorative analysis and thus, some groundwork in developing a system that would be able to generate indicative tweets from referenced documents, possibly based on parameters for it to be specifically an advertisement or a public announcement and so on.

With indicative tweets, an additional issue arises in that the genre of the source text is not constrained; for example it may be a news article or an informal blog post or an advertisement. This may be vastly different from the desired formality of tweet itself, and thus, a genre-appropriate extract may not be available.

We begin to address the above issues through a study that examines to what extent tweet generation can be viewed as an extractive summarization problem. We extracted a dataset of indicative tweets containing a link to an external article, including the documents linked to by the tweets. The process is detailed in \chapref{chap:data}. In \chapref{chap:analysis}, we used this data and applied unigram, bigram and LCS (longest common subsequence) matching techniques inspired by ROUGE to determine what proportion of tweets can be found in the linked article. Even with the permissive unigram match measure, we find that well under half of the tweet can be found in the linked article. We also use stylistic analysis on the articles to examine the role that genre differences between the source text and the target tweet play, and find that it is easier to extract tweets from more formal articles than less formal ones. 

We further conducted studies to identify functions for the tweets with respect to the articles in \chapref{chap:user}. The data extracted from Twitter was presented to workers on a crowdsourcing website, to ask for the indicativeness and the informativeness of the tweets. We found a link between whether the tweet was deemed informative, and the degree to which a tweet has been extracted from the article, offering a better view of our analysis methods and when they can be used for generating tweets. As a result, this dataset tagged by human evaluators has been generated and should be useful in further studies for identifying functions of tweets.

Overall, our results point to the need for the development of a methodology for indicative tweet generation, rather than to expropriate the extractive summarization paradigm that was developed mostly on news text. Such a methodology will ideally be sensitive to stylistic factors as well as the underlying intent of the tweet.

Portions of \chapref{chap:intro}, \chapref{chap:data} and \chapref{chap:analysis} were published in as part of the conference paper \cite{sidhayeindicative}. Contribution of the co-author was that of thesis supervisor. 
