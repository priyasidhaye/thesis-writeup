\chapter{Introduction}
\label{chap:intro}

Social media comprise a large part of our lives, with various outlets providing a platform for sharing thoughts, news, images and videos. With the rise in popularity of social media, message broadcasting sites such as Twitter and other microblogging services have become an important means of communication, with an estimated 500 million tweets being written every day\footnote{https://about.twitter.com/company}. In addition to individual users, various organizations and public figures such as newspapers, government officials and entertainers have established themselves on social media in order to disseminate information or promote their products. Social media thus provide an incredibly dense and varied source of data, originating from people and organizations all over the world.

Following the prolific increase in the use of social media, there has been an increase in the number of studies in natural language processing using the sources of data social media provide. Specifically for Twitter, these include areas such as tweet parsing \citep{ritter2011named, kong-etal-2014}, text normalization \citep{han2011lexical, kaufmann2010syntactic}, tweet POS tagging \citep{gimpel2011part, owoputi-etal-2013}, sentiment analysis of tweets \citep{kouloumpis2011twitter, mohammad2013nrc}, event summarization \citep{chakrabarti2011event, nichols2012summarizing}, identifying bot behaviours \citep{chu2012detecting} and inferring things like political views of individuals \citep{mohammad2013identifying}. While this progress in the development of Twitter-specific POS taggers, parsers, and other tweet understanding tools is encouraging, there has been little work on methods for \textit{generating tweets}. Methods to generate tweets would be beneficial to users and organizations for the purposes of advertisement, education, or even entertainment. Examples of such uses include generating tweets that advertise products or services based on some online review articles, or notifying users of closure of roads because of construction work by local governments.

% , despite the utility this would have for users and organizations, for purposes of advertisement, education, or even entertainment. 

In this thesis, we study the generation of the particular class of tweets that contain a link to an external web page that is composed primarily of text. Given the short length of a tweet, the presence of a URL in the tweet is a strong signal that the tweet is functioning to help Twitter users decide whether to read the full article. We call this class of tweets \emph{indicative tweets}, since they act as indicative summaries of the articles they are being linked to. Indicative tweets represent a large subset of tweets overall, constituting more than half (53.4\%) of the tweets in a data set that we collected in \chapref{chap:data}. Generating indicative tweets would appear to be a feasible problem to solve using current methods in text summarization, such as extractive summarization, because there is a clear source of input from which a tweet could be generated. 

% It should be noted that a tweet being indicative does not preclude it from also providing a critical evaluation of the linked article.

There has in fact been some work along these lines, within the framework of extractive summarization. \cite{lofi2012iparticipate} describe a system to generate tweets from local government records through keyword generation. However, they do not provide a formal evaluation for their proposed system.

\cite{lloret2013towards} compare various extractive summarization algorithms applied on Twitter data to generate tweets from documents. They compare the overlap between system-generated and user-generated tweets using ROUGE 
\citep{lin2004rouge}, a recall-based evaluation metric for summarization, and achieve some success in generating tweets based on ROUGE scores. Unfortunately, they also show that there is little correlation between ROUGE scores and the perceived quality of the tweets when rated by human users for indicativeness and interest. An \textit{indicative} text is one that aims to point to or generate interest about something. Hence the \textit{indicativeness} of a tweet can be defined as a measure of how strongly it points to the article. More discussion about these studies is done in \chapref{chap:background}.

Beyond issues of evaluation measures, it is also unclear whether extraction is the strategy employed by human tweeters. One of the original motivations behind extractive summarization for news text was the observation that human summary writers tended to extract snippets of key phrases from the source text \citep{mani-2001}.  And while it may be true that an automatic tweet generation system need not necessarily follow the same approach to writing as human tweeters, it is still necessary to know what proportion of tweets could be accounted for in an extractive summarization paradigm. More scrutiny is required to determine whether methods and evaluation schemes from extractive summarization can be adopted for the purpose of producing indicative tweets and is one of the primary aims of this thesis. With indicative tweets, an additional issue arises in that the genre of the source text is not constrained; for example it may be a news article or an informal blog post or an advertisement. This genre of the source text may be vastly different from the desired formality of tweet itself, and thus, a genre-appropriate extract may not be available. 


\paragraph{Contributions}
We begin to address the above issues through a study that examines to what extent tweet generation can be viewed as an extractive summarization problem. We extracted a dataset of indicative tweets containing a link to an external article, including the documents linked to by the tweets. We used this data and applied unigram, bigram and LCS (longest common subsequence) matching techniques inspired by ROUGE to determine what proportion of tweets can be found in the linked article.  This measure can also be defined as \emph{extractiveness} of the tweet, or the degree to which the tweet has been extracted from the article. Even with the permissive unigram match measure, we find that well under half of the tweet can be found in the linked article. We also use stylistic analysis on the articles to examine the role that genre differences between the source text and the target tweet play and if genre can give an indication for whether the tweet can be extracted. We find that tweets are extracted from articles with higher formality to a greater extent than ones with lower formality. 

We further conducted studies to identify functions for the tweets with respect to the articles. The data extracted from Twitter was presented to workers on a crowdsourcing website, to ask whether the tweets were indicative or informative. \textit{Informative} tweets are the ones that convey some information from the article, and \textit{informativeness} is defined as the degree to which the information from the article is conveyed in the tweet. We found a link between whether the tweet was deemed informative, and the degree to which a tweet has been extracted from the article, offering a better view of our ROUGE-inspired analysis methods detailed in \chapref{chap:analysis} and when they can be used for generating tweets. As a result, this dataset tagged by human evaluators has been generated and should be useful in further studies for identifying functions of tweets. 

Overall, our results point to the need for the development of a methodology for indicative tweet generation, rather than to expropriate the extractive summarization paradigm that was developed mostly on news text. Such a methodology will ideally be sensitive to stylistic factors as well as the underlying intent of the tweet.

\paragraph{Chapter Outline} \chapref{chap:background} contains the discussion on various related studies. The process of collecting the dataset is detailed in \chapref{chap:data}. \chapref{chap:analysis} contains the analyses performed on the dataset we collected. \chapref{chap:user} details the process of designing and executing the user study based on our data and analysis of the input from the workers. Finally, \chapref{chap:conclusion} contains the conclusions drawn from our the discussion of results in \chapref{chap:analysis} and \chapref{chap:user}.

Portions of \chapref{chap:intro}, \chapref{chap:data} and \chapref{chap:analysis} were published as part of the conference paper \cite{sidhayeindicative}. The contribution of the co-author was that of a thesis supervisor. 
